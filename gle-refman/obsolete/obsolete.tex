%%%%%%%%%%%%%%%%%%%%%%%%%%%%%%%%%%%%%%%%%%%%%%%%%%%%%%%%%%%%%%%%%%%%%%%%
%                                                                      %
% GLE - Graphics Layout Engine <http://www.gle-graphics.org/>          %
%                                                                      %
% Modified BSD License                                                 %
%                                                                      %
% Copyright (C) 2009 GLE.                                              %
%                                                                      %
% Redistribution and use in source and binary forms, with or without   %
% modification, are permitted provided that the following conditions   %
% are met:                                                             %
%                                                                      %
%    1. Redistributions of source code must retain the above copyright %
% notice, this list of conditions and the following disclaimer.        %
%                                                                      %
%    2. Redistributions in binary form must reproduce the above        %
% copyright notice, this list of conditions and the following          %
% disclaimer in the documentation and/or other materials provided with %
% the distribution.                                                    %
%                                                                      %
%    3. The name of the author may not be used to endorse or promote   %
% products derived from this software without specific prior written   %
% permission.                                                          %
%                                                                      %
% THIS SOFTWARE IS PROVIDED BY THE AUTHOR "AS IS" AND ANY EXPRESS OR   %
% IMPLIED WARRANTIES, INCLUDING, BUT NOT LIMITED TO, THE IMPLIED       %
% WARRANTIES OF MERCHANTABILITY AND FITNESS FOR A PARTICULAR PURPOSE   %
% ARE DISCLAIMED. IN NO EVENT SHALL THE AUTHOR BE LIABLE FOR ANY       %
% DIRECT, INDIRECT, INCIDENTAL, SPECIAL, EXEMPLARY, OR CONSEQUENTIAL   %
% DAMAGES (INCLUDING, BUT NOT LIMITED TO, PROCUREMENT OF SUBSTITUTE    %
% GOODS OR SERVICES; LOSS OF USE, DATA, OR PROFITS; OR BUSINESS        %
% INTERRUPTION) HOWEVER CAUSED AND ON ANY THEORY OF LIABILITY, WHETHER %
% IN CONTRACT, STRICT LIABILITY, OR TORT (INCLUDING NEGLIGENCE OR      %
% OTHERWISE) ARISING IN ANY WAY OUT OF THE USE OF THIS SOFTWARE, EVEN  %
% IF ADVISED OF THE POSSIBILITY OF SUCH DAMAGE.                        %
%                                                                      %
%%%%%%%%%%%%%%%%%%%%%%%%%%%%%%%%%%%%%%%%%%%%%%%%%%%%%%%%%%%%%%%%%%%%%%%%

\section{GLE Installation on a PC}
\index{DOS}
To install GLE, put the distribution disk into drive a: and
type:
\begin{verbatim}
     a:install
\end{verbatim}

GLE requires at least 800K of disk space for a minimal installation
and 3.3M for a full installation (including all device drivers
and fonts).

GLE also requires 530K or more of free memory. Use the DOS
command CHKDSK to check this figure. If you don't have
enough then take copies of your AUTOEXEC.BAT and CONFIG.SYS files
and then remove as much as possible from these files.

GLE may work with less memory depending on what you are drawing.

The installation disk contains a version of CGLE which will
make use of epanded memory.  This version can run with 70K less
memory but if you don't have expanded memory then it has to
use your hard disk instead which is a great deal slower.

\index{EPS}
\index{Encapsulated PostScript}
To produce an .eps file for inclusion in WordPerfect you would type:
\begin{verbatim}
     C:> psgle myfile /eps
\end{verbatim}
This will create a file called myfile.EPS.

WordPerfect can also understand a special file format which contains
both an EPS file for printing to PostScript printers and a bitmap  TIFF
image for displaying on the screen.  GLE can create this sort of
file with the command:
\begin{verbatim}
     C:> wpgle myfile
\end{verbatim}
This will create a file called MYFILE.EPF which you can include into
WordPerfect.

If your PC is connected to a VAX computer which has a PostScript printer
you may copy MYFILE.PS to the VAX using a standard file transfer
program (e.g. FTP, KERMIT, VDISK)

The best way to see what GLE can do is to have a play with it, simply
start it up and try out some of the examples:
\begin{verbatim}
     Press F3       (Load file)
     Press <enter>  (for a menu of GLE files)
     Use arrow keys to select example, then press <enter>
     Press F10 to draw the picture
     Press ESC to get back to the GLE editor.
     ...
     When you find a graph try pressing F9 and modifying
     one of the fields (use F1 for an explanation of each
     field).
\end{verbatim}

Used symbols for PC:
\begin{verbatim}
GLE_NOCONTROLD TRUE  (Stops ^D being added to ps files)
(this is the default for unix)

GLE_ADDBGI ...       (high resolution BGI-Driver)
(see Device Drivers)
\end{verbatim}
%
%
%%%%%%%%%%%%%%%%%%%%%%%%%%%%%%%%%%%%%%%%%%%%%%%%%%%%%%%%%%%%%%%%%%%%%%%%
%                                                                      %
% GLE - Graphics Layout Engine <http://www.gle-graphics.org/>          %
%                                                                      %
% Modified BSD License                                                 %
%                                                                      %
% Copyright (C) 2009 GLE.                                              %
%                                                                      %
% Redistribution and use in source and binary forms, with or without   %
% modification, are permitted provided that the following conditions   %
% are met:                                                             %
%                                                                      %
%    1. Redistributions of source code must retain the above copyright %
% notice, this list of conditions and the following disclaimer.        %
%                                                                      %
%    2. Redistributions in binary form must reproduce the above        %
% copyright notice, this list of conditions and the following          %
% disclaimer in the documentation and/or other materials provided with %
% the distribution.                                                    %
%                                                                      %
%    3. The name of the author may not be used to endorse or promote   %
% products derived from this software without specific prior written   %
% permission.                                                          %
%                                                                      %
% THIS SOFTWARE IS PROVIDED BY THE AUTHOR "AS IS" AND ANY EXPRESS OR   %
% IMPLIED WARRANTIES, INCLUDING, BUT NOT LIMITED TO, THE IMPLIED       %
% WARRANTIES OF MERCHANTABILITY AND FITNESS FOR A PARTICULAR PURPOSE   %
% ARE DISCLAIMED. IN NO EVENT SHALL THE AUTHOR BE LIABLE FOR ANY       %
% DIRECT, INDIRECT, INCIDENTAL, SPECIAL, EXEMPLARY, OR CONSEQUENTIAL   %
% DAMAGES (INCLUDING, BUT NOT LIMITED TO, PROCUREMENT OF SUBSTITUTE    %
% GOODS OR SERVICES; LOSS OF USE, DATA, OR PROFITS; OR BUSINESS        %
% INTERRUPTION) HOWEVER CAUSED AND ON ANY THEORY OF LIABILITY, WHETHER %
% IN CONTRACT, STRICT LIABILITY, OR TORT (INCLUDING NEGLIGENCE OR      %
% OTHERWISE) ARISING IN ANY WAY OUT OF THE USE OF THIS SOFTWARE, EVEN  %
% IF ADVISED OF THE POSSIBILITY OF SUCH DAMAGE.                        %
%                                                                      %
%%%%%%%%%%%%%%%%%%%%%%%%%%%%%%%%%%%%%%%%%%%%%%%%%%%%%%%%%%%%%%%%%%%%%%%%

\section{32bit DOS Version of GLE}
\index{DOS!32bit}
Axel Rohde has compiled a 32bit DOS version of GLE in 1994 
( email: rohde@physik.uni-kiel.d400.de)\\
On any 386 or better machine this version of GLE should 
run without problems, it's main features are these:
1) No 640K memory restrictions. 2) Much faster.\\
\clearpage
Properties of GLE 32:
\begin{verbatim}
 - All programs are running in the 386-protected-mode and therefore there is
   neither a limited 640kB adress-range nor a 64kB segmentation.
 - 32-bit programs are running faster than their 16-bit-counterparts.
 - There exists a multitude of GRX-graphics-drivers, e.g. for
   TSENG ET4000(W32), S3, 8514A, Cirrus Logic GD 542x, Trident 8900,
   Diamond Viper, ATI Ultra, ATI VGA and EGA. These drivers are highly
   configureable and can use flicker-free high resolution modes.
\end{verbatim}
Installation Quick guide:
\begin{verbatim}
    1) FTP the binary distribution
           ftp tui.marc.cri.nz
           cd pub/gle/gle32
           binary
           mget gle32bi*.zip
    2) Unzip them keeping the directory structure
           cd c:\
           pkunzip gle32bi1.zip -d
           pkunzip gle32bi2.zip -d
           pkunzip gle32bi3.zip -d
           pkunzip gle32bi4.zip -d
           pkunzip gle32bi5.zip -d
           pkunzip gle32bi6.zip -d
    3) Edit the batch file which tells gle where to find it's fonts
       and also what sort of graphics card you have.  
           edit setgle32.bat
       (change the disk and directory as appropriate)
    4) Run the batch script
           setgle32
    5) Try out the new version
            gle_vga
    6) Note most of the programs have been renamed to avoid conflicts!!!
\end{verbatim}

To avoid name-conflicts between a 16-bit and a 32-bit version of GLE, 
all the programs and the environment-variables were renamed. 
All GLE-programs have now unix-style names like \verb#gle_ps# (='psgle'-
Postscript-output), \verb#gle_vga# (='cgle' - VGA-Preview) etc. The names 
of the utilities end with '32' -  \verb#manip32#, \verb#contou32# \dots

Restrictions and Bugs:
\begin{description}		
\item[1.] The vector-fonts of the 32-bit-version are NOT compatible to their 
16-bit-counterparts. They may be compatible to fonts that were created 
under other 32-bit operating-systems.
\item[2.] The on-line-help of \verb#gle_vga# is usable but may sometimes look different 
compared to the original.
\item[3.] Makefmt and fbuild are missing. DJGPP's Library lacks ecvt(). 
Both programs are used to calculate vector-fonts in the Unix-version 
from the source-distribution. Both programs are NOT included in the 
16-bit DOS-version, too. This package includes all (allready calculated) 
fonts from the Unix-source distribution. They were calculated under Linux, 
a free Unix-implementation for i386-PC's and higher.
\item[4.] The DVI-drivers are not testet! 
\item[5.] Surface (\verb#surf_vga#) hanged under unknown circumstances while loading 
one data-file. \verb#surf_vga# can be stopped by pressing Control-Pause. 
If this happens, load the data-file into an editor, save it, and try it 
again.
\end{description}

In \verb#gle32b.txt# are the more detailed instructions provided by Axel Rohde.
Read these carefully if you have any problems.

%%\section{32bit DOS Version of GLE}
%%\index{DOS!32bit}
%

\section{Running GLE on a VAX}
\index{VAX}
The command to run GLE is:
\begin{verbatim}
     $ cgle myfile.gle
     $ cgle myfile.gle /dev=regis
     $ cgle myfile.gle /dev=x
\end{verbatim}
See the directory CGLE\_EXAMPLES: for examples and templates.
To get access to these files from the GLE menus use the commands:
\begin{verbatim}
     $ define workarea sys$login:,cgle_examples:
     $ set default workarea:
     $ cgle stack4b.gle      ! or any other example
\end{verbatim}

If your keyboard doesn't have the function keys F9 thru F14 you can
use GOLD (PF1) followed by the numbers 9,0,1,2,3,4 (on the top of the
QWERTYUIOP keypad).

Keyboard Mappings:\\
\begin{center}
\begin{tabular}{|l|l|l|l|} \hline
VT100  & VT200 &  PC	& Meaning	\\ \hline
GOLD 1 & F11  & F1	& Help 		\\
GOLD 2 & F12  & F2	& Save		\\
GOLD 3 & F13  & F3	& Load		\\
GOLD 4 & F14  & F4	& Save-as	\\
GOLD 9 & F9   & F9   	& Graph-menu	\\
GOLD 0 & F10  & F10  	& Draw-it 	\\
Control+Z & Control+Z & Control+Z 	& Exit/Escape \\
& & Alt+X		& Exit/Escape \\
Control+E & Control+E  & 	& Calls VAX EDT	\\
Control+F & Control+F  & 	& Toggle fast/slow text \\
Control+R & Control+R  & F5	& Shows errors \\
& & Control+S	& Shells to DOS \\ \hline
\end{tabular}
\end{center}

Supported devices: VT100, REGIS (VT125, VT240), TEK4010, VWS, XWindows.

Supported output: PostScript, HPGL, Epson, Epson 24pin, HP Deskjet.

\index{printing}
To create a PostScript output file (.PS) and automatically print it to
the LASER queue you would type:
\begin{verbatim}
     $ cgle myfile /print
\end{verbatim}

\index{EPS}
\index{Encapsulated PostScript}
To produce an .eps file for inclusion in \LaTeX\ you would type:
\begin{verbatim}
     $ cgle myfile /dev=eps
\end{verbatim}

To produce a .ps file suitable for printing to a laser writer type:
\begin{verbatim}
     $ cgle myfile /dev=ps
\end{verbatim}

The DCL symbol GLE\_NOCONTROLD should be set to "TRUE" to kill the
\verb#^D# on vms systems.

If you set the DCL symbol GLE\_EDITOR to TPU then you will get
TPU instead of EDT when you press \verb#^E#.

\section{It didn't work, bugs!!!}
\index{bugs} \index{errors}
If the installation fails, or one of the example GLE files fails to work
then the most likely reason is a shortage of memory due to too many
memory resident programs/drivers.  To fix this remove these utilities from
your autoexec.bat and config.sys files temporarily.

There may well be a bug in your GLE file,  try using the
trace option to find the bug.\\
On a PC:
\begin{verbatim}
     C:\GLE> psgle myfile /trace
\end{verbatim}

On a VAX:
\begin{verbatim}
     $ cgle myfile /dev=ps /trace
\end{verbatim}

Another reason for a failure is a bug in GLE,  {\bf Please} report
bugs to Chris Pugmire (Internet srghcxp@grv.dsir.govt.nz,  grv::srghcxp)
so they can be fixed. If possible, try and find a way of
repeating the problem, then send relevant GLE and data files with
an outline of what is wrong.

\index{device drivers}
\index{drivers} \index{EPSON} \index{LaserJet} \index{DeskJet}
\index{plotters} \index{printers} 
\section{Device Drivers}
GLE supports the following devices.

Interactive: IBM/PC (BGI), VT100, REGIS (VT125, VT240), TEK4010, VWS, XWindows.

Output: PostScript, HPGL, EPSON, EPSON 24pin, HP Deskjet.

Keyboard Mappings:
\begin{center}
\begin{tabular}{|l|l|l|l|} \hline
VT100  & VT200 &  PC	& Meaning	\\ \hline
GOLD 1 & F11  & F1	& Help 		\\
GOLD 2 & F12  & F2	& Save		\\ 
GOLD 3 & F13  & F3	& Load		\\
GOLD 4 & F14  & F4	& Save-as	\\
GOLD 9 & F9   & F9   	& Graph-menu	\\
GOLD 0 & F10  & F10  	& Draw-it 	\\ 
Control+Z & Control+Z & Control+Z 	& Exit/Escape \\
& & Alt+X		& Exit/Escape \\
Control+E & Control+E  & 	& Calls VAX EDT	\\
Control+F & Control+F  & 	& Toggle fast/slow text \\ \hline
\end{tabular}
\end{center}
To find out what drivers are available type in:
\begin{verbatim} 
     ls /usr/local/gle/gle_*
\end{verbatim}
\subsection{PC Screen Drivers}
\index{PC screen driver}
\index{Device Drivers!PC screen driver}
Remember that what you see on the screen isn't always what you will
get on the printer.  For example filled regions will not be filled,
and some characters may not look right.

After pressing F10 and drawing the graph it can be annotated by using 
the mouse (or arrow keys) to draw lines, text and boxes.  
To draw lines simply click on the points of the line, use the right
hand mouse button to `pick up' the pen.  

To draw text press the letter `t' and then click on where you would
like the text to be drawn.  

All movements are rounded to the grid size settings which are 
1.0cm, 0.1cm, 0.01cm etc.  

The height and colour of the text/lines is determined by the current
settings at the end of the GLE file.

If there is no mouse driver loaded then a cross-hair will appear 
and it can be moved around using the arrow keys.  Press `c' to 
click, instead of the mouse button.
% sm
\subsubsection{SuperVGA}
\index{Device Drivers!PC SVGA}
%
To use a SuperVGA card you first need to get a BGI
driver that supports your card.  From anonymous FTP
you can get svgabgi3.zip: ftp wuarchive.wustl.edu
\begin{verbatim}
     ftp> user anonymous
     ftp> (mail ident)
     ftp> cd /mirrors/msdos/borland
     ftp> binary
     ftp> get svgabgi3.zip
     ftp> quit
\end{verbatim}
Then unzip it, decide which driver will match your SVGA card
and copy that driver into \verb?\GLE\EXE?.
\begin{verbatim}
     c:> pkunzip svgabgi3.zip
     c:> pkunzip svgabgi3.zip
     c:> type readme. 
     c:> copy svga16.bgi \gle\exe
\end{verbatim}
Define an environment variable to tell gle about this driver
and which mode to use. (put this line in your autoexec.bat)
\begin{verbatim} 
     c:> set gle_addbgi=4.svga16
\end{verbatim}                                                                                      
% sm
%%\subsection{GO32/GRX Screen Driver}
%%%%%%%%%%%%%%%%%%%%%%%%%%%%%%%%%%%%%%%%%%%%%%%%%%%%%%%%%%%%%%%%%%%%%%%%%%
%                                                                      %
% GLE - Graphics Layout Engine <http://www.gle-graphics.org/>          %
%                                                                      %
% Modified BSD License                                                 %
%                                                                      %
% Copyright (C) 2009 GLE.                                              %
%                                                                      %
% Redistribution and use in source and binary forms, with or without   %
% modification, are permitted provided that the following conditions   %
% are met:                                                             %
%                                                                      %
%    1. Redistributions of source code must retain the above copyright %
% notice, this list of conditions and the following disclaimer.        %
%                                                                      %
%    2. Redistributions in binary form must reproduce the above        %
% copyright notice, this list of conditions and the following          %
% disclaimer in the documentation and/or other materials provided with %
% the distribution.                                                    %
%                                                                      %
%    3. The name of the author may not be used to endorse or promote   %
% products derived from this software without specific prior written   %
% permission.                                                          %
%                                                                      %
% THIS SOFTWARE IS PROVIDED BY THE AUTHOR "AS IS" AND ANY EXPRESS OR   %
% IMPLIED WARRANTIES, INCLUDING, BUT NOT LIMITED TO, THE IMPLIED       %
% WARRANTIES OF MERCHANTABILITY AND FITNESS FOR A PARTICULAR PURPOSE   %
% ARE DISCLAIMED. IN NO EVENT SHALL THE AUTHOR BE LIABLE FOR ANY       %
% DIRECT, INDIRECT, INCIDENTAL, SPECIAL, EXEMPLARY, OR CONSEQUENTIAL   %
% DAMAGES (INCLUDING, BUT NOT LIMITED TO, PROCUREMENT OF SUBSTITUTE    %
% GOODS OR SERVICES; LOSS OF USE, DATA, OR PROFITS; OR BUSINESS        %
% INTERRUPTION) HOWEVER CAUSED AND ON ANY THEORY OF LIABILITY, WHETHER %
% IN CONTRACT, STRICT LIABILITY, OR TORT (INCLUDING NEGLIGENCE OR      %
% OTHERWISE) ARISING IN ANY WAY OUT OF THE USE OF THIS SOFTWARE, EVEN  %
% IF ADVISED OF THE POSSIBILITY OF SUCH DAMAGE.                        %
%                                                                      %
%%%%%%%%%%%%%%%%%%%%%%%%%%%%%%%%%%%%%%%%%%%%%%%%%%%%%%%%%%%%%%%%%%%%%%%%

\subsection{GO32/GRX Screen Driver}
\index{GLE32!GRX}
\index{GO32}
\index{GLE32!SVGA}
GLE 32 is a 32-bit-Implementation of GLE for MS-DOS. 
Axel Rohde did this port of the GLE version 3.3b 
GLE32 was compiled with the free port 
of the GNU C-compiler DJGPP by D.J. Delorie. The compiler itself and 
the DJGPP-compiled executables are running with the 32-bit DOS-extender 
GO32.\\
%
GLE32 was compiled using Borland-C compatible libraries for text and 
graphics-modes and some filesystem calls. 
There exists for the DJGPP specific graphics-library GRX from Csaba Biegl 
a emulation-library called BCCGRX from Hartmut Schirmer 
to replace calls of the Borland-Graphics-Interface (BGI) with GRX-calls. 
Hartmut has implemented the mouse-functionality, too, with GRX-calls.\\
%
In the example on the following lines, the progams are installed in the 
directory \verb#d:\gle32#, the fonts are in \verb#d:\gle32\fonts#. 
GLE 32 searches for
its vector-fonts in the directory \verb#GLE_TOP#. Don't forget the trailing slash. 
In addition to this, \verb#GL32FONT# points to the directory where the bitmap-fonts 
(you can see them in the status-line of the preview) can be found. \\
%
The directory with the DOS-Extender GO32 and, if there's no 
numeric-prozessor installed, the co-prozessor-emulator, must be in the 
path-environment. 
\begin{verbatim}
    set GLE_TOP=d:/gle32/
    set gle32font=d:/gle32/grxfont
    path=..your normal path..;d:\gle;
    go32=driver d:/gle32/driver/vesa_s3.grn gw 1024 gh 768 tw 80 th 25 nc 256 
\end{verbatim}


The configuration of graphics-drivers is a little bit more complicated. 
Please study the documentation of the Libraries GRX und BCCGRX and 
the README of GO32 in their directories. 

The environment GO32 sets den path-name and the mode of the driver 
an. This example installs the driver for an S3 graphics-board with a 
resolution of 1024 horizontal 768 vertical pixels and 256 colors. 

There's a second way to install a graphics-mode. If the environment-
variables GLE32WIDTH, GLE32HEIGHT and GLE32COLORS are set, the graphics-
mode in the GO32 variable is overridden. You still have to specify a 
driver in the GO32 environment.
\begin{verbatim}   
    set GLE32WIDTH=800
    set GLE32HEIGHT=600
    set GLE32COLORS=16
\end{verbatim}

There's a prepared batch-file \verb#setgle32.bat# to set all the environment-
variables in the directory gle32.

To figure out, which modes are supported, try to run the program
modetest in the directory \verb#gle32\driver\doc#.\\

\begin{center} ! WARNING !\end{center}

A wrong installed graphics-mode can DESTROY your MONITOR (and/or graphics 
board) if the refresh rate is too high!

\begin{center} USE THIS ON YOUR OWN RISK ! \end{center}

You should take a look into the manuals of your monitor and your 
graphics-adapter to figure out, which horizontal frequencies are supported. 
If the horizontal frequency of the monitor is 64 kHz or higher, 
you MAY feel save. (MY Monitor has a horizontal frequency of 64 kHz and 
supports a resolution of 1024x768 with a refresh rate up to 80Hz.)\\
% 
If you want to go save, don't set the environment go32 at all! 
This will use the normal (flickering) VGA-mode on a VGA-compatible 
adapter.\\
%
For more detailed information and instructions how to install GO32 see
\verb#gle32.txt#!

%%%%%%%%%%%%%%%%%%%%%%%%%%%%%%%%%%%%%%%%%%%%%%%%%%%%%%%%%%%%%%%%%%%%%%%%
%                                                                      %
% GLE - Graphics Layout Engine <http://www.gle-graphics.org/>          %
%                                                                      %
% Modified BSD License                                                 %
%                                                                      %
% Copyright (C) 2009 GLE.                                              %
%                                                                      %
% Redistribution and use in source and binary forms, with or without   %
% modification, are permitted provided that the following conditions   %
% are met:                                                             %
%                                                                      %
%    1. Redistributions of source code must retain the above copyright %
% notice, this list of conditions and the following disclaimer.        %
%                                                                      %
%    2. Redistributions in binary form must reproduce the above        %
% copyright notice, this list of conditions and the following          %
% disclaimer in the documentation and/or other materials provided with %
% the distribution.                                                    %
%                                                                      %
%    3. The name of the author may not be used to endorse or promote   %
% products derived from this software without specific prior written   %
% permission.                                                          %
%                                                                      %
% THIS SOFTWARE IS PROVIDED BY THE AUTHOR "AS IS" AND ANY EXPRESS OR   %
% IMPLIED WARRANTIES, INCLUDING, BUT NOT LIMITED TO, THE IMPLIED       %
% WARRANTIES OF MERCHANTABILITY AND FITNESS FOR A PARTICULAR PURPOSE   %
% ARE DISCLAIMED. IN NO EVENT SHALL THE AUTHOR BE LIABLE FOR ANY       %
% DIRECT, INDIRECT, INCIDENTAL, SPECIAL, EXEMPLARY, OR CONSEQUENTIAL   %
% DAMAGES (INCLUDING, BUT NOT LIMITED TO, PROCUREMENT OF SUBSTITUTE    %
% GOODS OR SERVICES; LOSS OF USE, DATA, OR PROFITS; OR BUSINESS        %
% INTERRUPTION) HOWEVER CAUSED AND ON ANY THEORY OF LIABILITY, WHETHER %
% IN CONTRACT, STRICT LIABILITY, OR TORT (INCLUDING NEGLIGENCE OR      %
% OTHERWISE) ARISING IN ANY WAY OUT OF THE USE OF THIS SOFTWARE, EVEN  %
% IF ADVISED OF THE POSSIBILITY OF SUCH DAMAGE.                        %
%                                                                      %
%%%%%%%%%%%%%%%%%%%%%%%%%%%%%%%%%%%%%%%%%%%%%%%%%%%%%%%%%%%%%%%%%%%%%%%%

\subsection{GO32/GRX Screen Driver}
\index{GLE32!GRX}
\index{GO32}
\index{GLE32!SVGA}
GLE 32 is a 32-bit-Implementation of GLE for MS-DOS. 
Axel Rohde did this port of the GLE version 3.3b 
GLE32 was compiled with the free port 
of the GNU C-compiler DJGPP by D.J. Delorie. The compiler itself and 
the DJGPP-compiled executables are running with the 32-bit DOS-extender 
GO32.\\
%
GLE32 was compiled using Borland-C compatible libraries for text and 
graphics-modes and some filesystem calls. 
There exists for the DJGPP specific graphics-library GRX from Csaba Biegl 
a emulation-library called BCCGRX from Hartmut Schirmer 
to replace calls of the Borland-Graphics-Interface (BGI) with GRX-calls. 
Hartmut has implemented the mouse-functionality, too, with GRX-calls.\\
%
In the example on the following lines, the progams are installed in the 
directory \verb#d:\gle32#, the fonts are in \verb#d:\gle32\fonts#. 
GLE 32 searches for
its vector-fonts in the directory \verb#GLE_TOP#. Don't forget the trailing slash. 
In addition to this, \verb#GL32FONT# points to the directory where the bitmap-fonts 
(you can see them in the status-line of the preview) can be found. \\
%
The directory with the DOS-Extender GO32 and, if there's no 
numeric-prozessor installed, the co-prozessor-emulator, must be in the 
path-environment. 
\begin{verbatim}
    set GLE_TOP=d:/gle32/
    set gle32font=d:/gle32/grxfont
    path=..your normal path..;d:\gle;
    go32=driver d:/gle32/driver/vesa_s3.grn gw 1024 gh 768 tw 80 th 25 nc 256 
\end{verbatim}


The configuration of graphics-drivers is a little bit more complicated. 
Please study the documentation of the Libraries GRX und BCCGRX and 
the README of GO32 in their directories. 

The environment GO32 sets den path-name and the mode of the driver 
an. This example installs the driver for an S3 graphics-board with a 
resolution of 1024 horizontal 768 vertical pixels and 256 colors. 

There's a second way to install a graphics-mode. If the environment-
variables GLE32WIDTH, GLE32HEIGHT and GLE32COLORS are set, the graphics-
mode in the GO32 variable is overridden. You still have to specify a 
driver in the GO32 environment.
\begin{verbatim}   
    set GLE32WIDTH=800
    set GLE32HEIGHT=600
    set GLE32COLORS=16
\end{verbatim}

There's a prepared batch-file \verb#setgle32.bat# to set all the environment-
variables in the directory gle32.

To figure out, which modes are supported, try to run the program
modetest in the directory \verb#gle32\driver\doc#.\\

\begin{center} ! WARNING !\end{center}

A wrong installed graphics-mode can DESTROY your MONITOR (and/or graphics 
board) if the refresh rate is too high!

\begin{center} USE THIS ON YOUR OWN RISK ! \end{center}

You should take a look into the manuals of your monitor and your 
graphics-adapter to figure out, which horizontal frequencies are supported. 
If the horizontal frequency of the monitor is 64 kHz or higher, 
you MAY feel save. (MY Monitor has a horizontal frequency of 64 kHz and 
supports a resolution of 1024x768 with a refresh rate up to 80Hz.)\\
% 
If you want to go save, don't set the environment go32 at all! 
This will use the normal (flickering) VGA-mode on a VGA-compatible 
adapter.\\
%
For more detailed information and instructions how to install GO32 see
\verb#gle32.txt#!


%
\subsection{PostScript Driver}
\index{PostScript (driver)}
\index{printing}
To print a GLE file to the laser printer type:
\begin{verbatim}
     $ cgle myfile /print
\end{verbatim}

or on a PC:
\begin{verbatim}
     C:\GLE> psgle myfile   
     C:\GLE> print myfile.ps 
\end{verbatim}

\index{flipping}
The postscript drivers for GLE will automatically flip a 
picture to best fit onto the page,  e.g. a wide graph
(as defined by the size command at the top) will be drawn 
in {\em landscape} mode and a tall thin graph will be drawn in
{\em portrait} mode.

\index{EPS}
\index{Device Drivers!PostScript}
\index{PostScript (EPS, WordPerfect)}
\index{Encapsulated PostScript}
\index{WordPerfect}
\index{\LaTeX \ (inserting graphs)}
To produce an .eps file on a VAX for inclusion in \LaTeX \ or 
WordPerfect you would type:
\begin{verbatim}
     $ cgle myfile /dev=eps
\end{verbatim}

On a PC you would type:
\begin{verbatim}
     C:\GLE> psgle myfile  /eps
     (this creates myfile.eps, not myfile.ps)
\end{verbatim}

WordPerfect can also understand a special file format which contains
both an EPS file for printing to PostScript printers and a bitmap  TIFF 
image for displaying on the screen.  GLE can create this sort of 
file with the command:
\begin{verbatim}
     C:> wpgle myfile
\end{verbatim}
This will create a file called MYFILE.EPF which you can include into
WordPerfect.

%Inside your \LaTeX \ document use the \LaTeX \  command:
%\begin{verbatim}
%     \graphin{myfile.eps}{12.0cm}{3.0cm}{1.0}
%\end{verbatim}
% sm
To include Postscript in a \LaTeX \ document use epsf.sty of the 
{\sf DVIPS}-package.  
Inside your \LaTeX \ document use {\sf\\epsffile } to put it at the right
place in a {\sf picture}-environment, e.g. :
\begin{verbatim}
     \documentstyle[12pt,epsf]{article}
     ...
     \begin{picture}(13.0, 5.5)
     \put(0.0,0.0){
     \epsfclipon
     \epsfxsize=6cm
     \epsffile[60 170 560 740]{myfile.ps} } % [ bounding-box ] { file_name }
     \end{picture}
\end{verbatim}
	
Advanced features of {\sf epsf.sty} are described in {\tt
../TeX/src/dvips/dvips.tex}.
% sm

\index{lwidth}
\index{line width}
The laser printer driver will draw all zero width lines .02cm wide
for any line width equal to zero, but if an lwidth 
is greater than zero and less than or equal to .0001 then it will use a line 
width of 1 pixel.  Without this it would be impossible to specify 
a line width that didn't occasionally get rounded to 2 pixels.

There are also some commandline switches
\begin{verbatim}
                /nod       (add ^D to ps file)
                /addd      (Don't add ^D to ps file)
                /nomaxpath (Don't choke on complex fill paths)
                /fill      (For dvibit, makes bar's filled instead of shaded)
\end{verbatim}

\index{Device Drivers!TEK4010}
\subsection{TEK4010 Driver}
This driver allows initialization sequences to be defined 
with the symbols TEK\_OPEN and TEK\_CLOSE.

On a VAX this is normally done by CGLECMD.COM so that when you
specify /DEV=V550 the assignments are done for you.
(V550 = Visual 550)

\index{plotter pens}
\index{pens} \index{HPGL}
\subsection{HPGL Driver}
\index{Device Drivers!HPGL Driver}
This driver allows initialization sequences to be defined 
in the symbols HPGL\_OPEN and HPGL\_CLOSE.  On the PC 
use environment variables and on the VAX use DCL symbols.
Also HPGL\_WIDTH and HPGL\_HEIGHT can be defined for non A3 plotters.

The HPGL driver supports sizes greater than A3.

The symbols HPGL\_ADDX, HPGL\_ADDY add a margin to the plot.  
These default to .9cm and 1.5cm

On a VAX this is normally done by CGLECMD.COM so that when you
specify /DEV=HPA4 the assignments are done for you.

The HPGL driver assigns the following colours to pen numbers:

1=black, 2=red, 3=green, 4=blue, 5=magenta, 6=white, 7=yellow.

To get Gle-Output in a WIN-WORD text, HPGL is a good and easy way to do it.
\index{Win-Word} \index{Win-Word!HPGL-Import}

The use of the HPGL-Converter HP2XX of Heinz W. Werntges is another way to
get various pix-map format output. HP2XX has been ported successfully 
for Unix, OS/2, DOS, Amiga, Atari, VMS and AXP. \\ 
HP2XX has many controling options. In
the current version it is able to write the following formats: 
mf, cad, em, epic, eps, pcl, pcx, pic, img, pbm, pre, uis.

HP2XX is available via anonymous FTP at ftp.rz.uni-duesseldorf.de. For
further information ask Heinz (Internet:
werntges@clio.rz.uni-duesseldorf.de).
\index{hp2xx}
\index{HPGL!converter}
%

\subsection{PC Bitmap Drivers}
\index{PC Bitmap Driver}
\index{Device Drivers!PC Bitmap Driver}
GLE supports the EPSON 8 and 24 pin and HP deskjet/laserjet printers.
To support bitmap devices which require a large amount of memory GLE 
first writes a device independent file OUT.DVI, then the
appropriate bitmap driver for your printer will read the OUT.DVI file
and create a bitmap which it then prints to LPT1:

\begin{verbatim}
     C:> dvigle myfile      (produces OUT.DVI)
     C:> dviepson           (creates bitmap and prints to LPT1:)
\end{verbatim}

The output options are:
\begin{verbatim}
     C:> dviepson           Standard EPSON printers 
     C:> dviep24            24 Pin EPSON printers (180dpi)
     C:> dvilj              HP Laser jet, Desk jet (150 dpi)
     C:> dvilj300           HP Laser jet, Desk jet (300 dpi)
     C:> dviprint -dpj      HP PaintJet printers
     C:> dviprint -dsx      Sixel graphics driver
\end{verbatim}
The high resolution drivers (dviep24, dvilj300) are significantly slower
than the low resolution drivers so would only be used for final output.

See the file AAREADME.GLE for the latest information on drivers.
% sm
\clearpage
\subsection{DVIPRINT}
\index{Bitmap Driver}
\index{Device Drivers! DVIPRINT}
All available bitmap drivers and options can be use with 
\begin{verbatim}
     gle myfile -ddevice [-hires] [...]
\end{verbatim} 
To find out what drivers are available type in:
\begin{verbatim}
     gle_dviprint

Usage: dviprint [-depson | -dlj] [-old] [-hires] [-vhires] [-debug] [-output xx.prt]
        -depson   To produce output for epson printers 
        -dwp      To add tiff image to wp .eps file 
        -dec      Epson color ribbon printers 
        -dlj      To produce output for HP LaserJet printers 
        -dpj      To produce output for HP PaintJet printers 
        -dsx      Sixel graphics, for DEC printers la100's ...
        -dvt      To print bitmap to screen (preview) 
        -over     Overhead transparency mode for PaintJet
        -old      For old HP Laser Jet printers (no compression)
        -hires    Uses high resolution for that printer (slower)
        -vhires   Viritcal high resolution for sixel graphics
        -wide     If your printer has a wide carriage 
        -noflip   Disable's auto flipping 
        -flip     Forces flip  
        -nosquash Tries to print it full size
        -compress Force internal bitmap compression (slow) 
        -noff     No form feed
        -debug    Prints debug messages
        -out x.x  Prints output to file instead of printer port
\end{verbatim}
% sm
% 
\subsection{Fonts (font mapping)}
\index{Device Drivers!fonts}
By default the generic fonts (rm, rmb, ss, tt etc) will all map
to PLSR (plotter simplex roman) on BITMAP and HPGL drivers.  To make
this happen on other drivers put the command 
{\sf plotter fonts} immediately after the {\sf size} command 
at the top of the GLE file.

A typical result of this change in fonts is that something 
that lines up on the screen will not line up when printed 
to an EPSON printer.  If this happens then use the {\sf plotter fonts}
command.

If a character is missing from a font, or isn't the particular 
variation you like, you can define a character to be from 
a different font in this way: 
\begin{verbatim}
     \chardef{%}{{\setfont{texcmr}\char{37}}} 
     \chardef{[}{{\setfont{texcmr}\char{91}}} 
     \chardef{]}{{\setfont{texcmr}\char{93}}} 
\end{verbatim}

	On the PC some fonts may not be installed to save disk space.
When one of the missing fonts is  called for, a replacement 
font will be displayed, this may look
terrible and some special characters may be completely wrong.

\clearpage
More importantly if you use a font which does not have its font metric
file installed (e.g. C:/GLE/FONTS/PLSR.FMT) then the
PostScript driver will space the characters incorrectly.  This can 
be fixed by extracting that particular metric file from the 
distribution file CGLE\_FVE.ZIP using PKUNZIP. 

\subsection{Printer Fonts}
\index{Device Printer!fonts}
You can now tell gle about postscript fonts that you have 
downloaded into your own printer.

Lets pretend we are adding a font called Greek-Bold
\begin{verbatim}
                1) Download the font into you laser printer
                2) Add a line to FONT.DAT
psgb 86 psgb.fmt plsr.fve psgb.fmt
a    b  c        d
                a = GLE's name for the font
                b = The next unused number in font.dat
                c = The font metric file, this can be created using
                        makefmt from and adobe font metric file
                        (.afm) which you should have been supplied
                        with your font.
                d = The font vector file, this is just a font that
                        that gle can use to draw your font on non
                        postscript devices.
                3) Add a line to psfont.dat
psgb Greek-Bold
\end{verbatim}
This tells the postscript driver that this is a 
font that the printer knows how to deal with.
%%%%%%%%%%%%%%%%%%%%%%%%%%%%%%%%%%%%5
